\chapter{PathObjects for Squeak/Smalltalk}
\label{c:implementation}
To validate the applicability of the \textsc{PathObjects} concept, we put it into practice with an implementation for the Squeak/Smalltalk environment.
This chapter gives an insight into implementation details and the peculiarities of the target environment that had to be considered.
Section \ref{s:ImplementationPrototype} illustrates the realization of the fundamental diagram components that are presented in the previous chapter.
Section \ref{s:ImplementationTracing} describes the necessary extensions of the step-wise run-time analysis approach that allow to reconstruct object interactions and referential relationships.
Finally, Section \ref{s:ImplementationLayouting} illustrates how \textsc{PathObjects} traces can be mapped to graphs, and how this circumstance is exploited to place objects and to route messages.

\section[Implementation of the Fundamental Diagram Components]{Implementation of the Fundamental Diagram Components%
\sectionmark{Implementation}}
\sectionmark{Implementation}
\label{s:ImplementationPrototype}

\begin{figure}[tb]
	\centering
	\includegraphics[width=1\textwidth]{../images/04-ImplMainWindow}
	\caption[PathObjects for Squeak/Smalltalk]{Implementation of \textsc{PathObjects} for Squeak/Smalltalk. 1) Diagram Area 2) Timeline 3) Mini-Map 4) Search Bar.}
	\label{fig:ImplementationMainWindow}
\end{figure}

Figure \ref{fig:ImplementationMainWindow} shows our implementation of \textsc{PathObjects} for the Squeak/Smalltalk environment.
The diagram depicts the same scenario which is covered in Figure \ref{fig:ApproachBasic}:
the analog clock class registers one of its instances as observer of the clock timer instance.

The example shows that the class \inlinecode{AnalogClock} sends the \inlinecode{attach:} message to an instance of \inlinecode{ClockTimer}.
Thereby, the green background color indicates objects that are participating in the current step of the execution history.
The argument of the \inlinecode{attach:} message is an instance of \inlinecode{AnalogClock}.
This circumstance is illustrated through the yellow role indicator which is attached to the object.
To clarify the role of an object in the current step even further, this indicator also shows the name the underlying method defines for this argument, namely \inlinecode{anObserver}.

\subsection{Interactive Components}
Developers can navigate the temporal dimension of execution traces through the timeline (cf. Figure \ref{fig:ImplementationMainWindow}, Component 2).
The spatial domain is navigable through the mini-map (Component 3) or through drag interactions with the diagram area.
The search bar (Component 4) enables target-oriented navigation and offers auto-completion for all object and message names of the current trace.

The icon to the left of message names provides entry points for the exploration of the underlying source code.
The implementations of the current message and of the context in which this message is sent are displayed in a source browser pop-up.
The icon to the right of message names allows to explore the state of the according arguments.

Each object features a context sensitive menu which allows to take further actions regarding the respective object.
Such a menu, which is displayed when developers hover over an object, is visible above the clock timer instance.
From left to right, it provides actions for toggling state tracking, toggling outbound reference tracking, setting of a quick filter, and execution of a quick search.

\begin{figure}[b]
	\centering
	
	\begin{subfigure}{0.45\textwidth}
		\centering
		\includegraphics[width=\textwidth]{../images/04-ImplExplorationState1}
		\caption[Object State Before Message Execution]{}
		\label{fig:ImplementationExplorationStateBefore}
	\end{subfigure}
	\quad
	\begin{subfigure}{0.45\textwidth}
		\centering
		\includegraphics[width=\textwidth]{../images/04-ImplExplorationState2}
		\caption[Object State After Message Execution]{}
		\label{fig:ImplementationExplorationStateAfter}
	\end{subfigure}
	
	\caption[Exploration of Object State Changes]{Exploration of object state changes. a) State of the clock timer instance before the execution of the \inlinecode{attach:} message. b) State of the same instance after the execution. The \inlinecode{observers} collection now includes the attached clock instance.}
	\label{fig:ImplementationExplorationState}
\end{figure}

Figure \ref{fig:ImplementationExplorationState} shows how enabling state exploration affects the information presentation.
In Figure \ref{fig:ImplementationExplorationStateBefore}, the state of the system is shown before the \inlinecode{attach:} message has been processed.
State tracking is enabled for the clock timer instance, and the state explorer shows that no observers are currently registered and that the hour, minute, and second attributes are equal to \inlinecode{0}.
Figure \ref{fig:ImplementationExplorationStateAfter} depicts the state of the system after the \inlinecode{attach:} message has been processed, or respectively after the developer took a step forward in the execution history.
The message label now has an additional field to explore the return value.
The latter turns out to be the clock timer instance, and this fact is also clarified through the role indicator that is attached to the object.
But the most important change is visible in the state explorer.
It can be seen that the collection of observers now has one entry, and this entry is the instance of \inlinecode{AnalogClock} which has been passed as argument.

\subsection{Information Layers}

\begin{figure}[tb]
	\centering
	\includegraphics[width=0.7\textwidth]{../images/04-ImplOverlays}
	\caption{Configuration Dialog for Information Layers.}
	\label{fig:ImplementationLayers}
\end{figure}

The dialog that allows to add and remove information layers is shown in figure \ref{fig:ImplementationLayers}.
New layers can be configured with the help of three selection boxes.
First, developers select a metric they wish to visualize. 
The metrics that are currently supported are listed in Table \ref{t:ApproachLayers}  and related to applicable diagram dimensions.
The second box is then populated with all diagram dimensions the selected metric can be mapped to.
If the selected dimension requires the definition of a color scale, this scale can be selected from the third list, which can be ignored otherwise.
In the depicted example, the developer chooses to display which methods have been changed last, and wishes to visualize this information with the help of the width of the edges that represent message sends.

Layers that are currently active are shown in the list filling the center of the dialog.
For instance, one user-defined layer maps the \textsc{Fanin}-metric to the color of objects on the mini-map.
Another layer maps the \textsc{Fanout}-metric to the background color of objects, and uses a different color scale for this purpose.
Examples how such layers affect the visualization of various diagram dimensions already have been given in the previous Chapter (cf. Figures \ref{fig:ApproachTimelineMetrics}, \ref{fig:ApproachMinimapMetric}).

When a developer chooses to visualize a metric through colors, the scale can be either threshold-based or linear.
In the first case, a traditional traffic light rating system is used ranging from red (metric indicates room for improvement) to green (metric indicates satisfactory conditions).
In the second case, the developer can choose from a range of linear and perceptually linearized color scales as proposed by Levkowitz et al. \cite{levkowitz_color_1992, levkowitz_color_1997}.

\section{Data Acquisition}
\label{s:ImplementationTracing}
When it comes to the acquisition of the required information about object-oriented program behavior, two challenges have to be tackled with our implementation of \textsc{PathObjects}.
First, the Squeak environment does not offer an adequate functionality to distinguish and identify objects during tracing runs.
Without such a functionality, it would be impossible to reconstruct object interactions from execution traces.
Consequently, a custom object identification strategy has to be implemented as extension of the underlying step-wise run-time analysis approach.
Second, the tracking and mapping of object references are required for the interactive reference exploration component.
However, the up-front collection of outbound object pointers would yield excessive data amounts.
Thus, this feature has to be implemented with the help of a refinement run strategy.

\subsection{Identification of Objects}
\label{ss:ImplementationTracingIdentification}
The fundamental requirement for the reconstruction of object interactions from traces is that object instances can be identified unambiguously.
Unfortunately, the Squeak environment does not provide such a functionality out of the box.
Though objects do have an \inlinecode{identityHash} attribute, the value of this attribute is neither unique nor stable \cite{goldberg_smalltalk-80:_1983}.
In the virtual machine we used, it is derived from the least significant bits of the object pointer which is managed by the virtual machine and represents the object's address in main memory.
Hence, the value of the object pointer and thus the value of the \inlinecode{identityHash} are subject to change during garbage collection, which may cause objects to be moved in main memory.
Furthermore, since the \inlinecode{identityHash} uses only the 12 least significant bits, there is no guarantee that two objects that are not referentially equal cannot share a common hash value \cite{goldberg_smalltalk-80:_1983}.

The Squeak image does not offer a convenient way to access the object pointer, which could serve as unique identifier for objects.
There are workarounds that would allow to determine its value, but since this value still is subject to change during a tracing run due to the reasons pointed out above, it would require to disable garbage collection during this time span.
This could have undesired side effects, like excessive memory consumption or slow execution speeds due to increased memory allocation effort.

For these reasons, we chose a different approach that allows us to generate unique, stable identifiers for objects.
It exploits the fact that a comparison for referential equality provided by the \inlinecode{==} method will always yield the correct result.
It returns true if two objects have the same object pointer, and false otherwise.
Squeak provides a \inlinecode{WeakKeyIdentityDictionary} that has two beneficial properties for our use case.
First, it uses \inlinecode{==} comparison to test dictionary keys for equality.
That means that when using the objects that occur during tracing runs as keys, it is guaranteed that no two referentially distinct objects will be identified as the same key, and an object will always be mapped to the same key once it is present in the dictionary.
The second property is that it uses weak references to manage keys, which implicates that garbage collection of an object is not suspended by the fact that it is used as key in such a dictionary.

Consequently, this allows us to use objects occurring during tracing runs as keys, without interfering with the test execution.
Identifiers can be assigned to objects as values of the dictionary.
Thus, it is guaranteed that one object will always be mapped to the same identifier.
To ensure that no identifier is assigned to two referentially different objects, it is sufficient to use a continuously incrementing integer counter when assigning new identities, or respectively when adding new entries to the dictionary.

Since we presuppose that test cases are strictly deterministic (cf. Section \ref{s:DiscussionLimitations}), this strategy results in another benefit.
Namely, objects will always be assigned the same identifier in different tracing runs, since the identifier is only dependent on the chronological order in which objects occur in a trace.
In other words, to stick to the observer example, the instance of \inlinecode{DigitalClock} will always be assigned with the identifier \inlinecode{27} no matter how often the test is executed, since it is the 27th object that occurs in the trace.
This is an important property and prerequisite for the tracing and mapping of outbound object pointers (cf. Section \ref{ss:ImplementationTracingReferences}).

\subsection{Extensions of the Tracing Framework}
\label{ss:ImplementationTracing}
In order to be suitable for the purposes of \textsc{PathObjects}, the tracing framework provided by \textsc{PathTools} had to be extended in two places.
First, a custom \inlinecode{MethodWrapper} had to be implemented that collects the information required for the reconstruction of object interactions.
For each wrapped executed method, this specialized wrapper reports integer identifiers of the receiving object, of the method arguments, and of the return value to the tracer.
Thereby, argument objects are flattened.
That means that if collections of objects are used as arguments, the identifiers of the objects contained in a collection are reported in addition to the identifier of the collection object itself.
Note that an identifier for the sender of a message is not captured, since it would require the expensive operation of stack inspection.
However, this information can be easily reconstructed from the tree structure of the traces, since the sender of a message is identical to the parent element of a call node in the call tree.

The second extension that had to be made is the implementation of a custom \inlinecode{Tracer} that provides some functionality required by our method wrappers and for additional refinement runs.
It provides an implementation of our object identification approach (cf. Section \ref{ss:ImplementationTracingIdentification}) that is used by the method wrappers to generate unique and stable identifiers for objects that occur during tracing.
Furthermore, it provides entry points for the exploration of outbound object references, and is responsible for the execution of refinement runs that are necessary to collect this information (cf. Section \ref{ss:ImplementationTracingReferences}).

\subsection{Reference Tracing}
\label{ss:ImplementationTracingReferences}
The exploration of object references is one of the interactive components of \textsc{PathObjects} diagrams.
However, similar to the exploration of object states, the up-front collection of this information would potentially invalidate the benefits of the step-wise run-time analysis approach.
First, the determination of the outbound object pointers of an object is time consuming and thus could slow down the tracing process significantly.
Second, the amount of collected data could grow rapidly if all outbound pointers of all involved objects were collected up-front.
For these reasons, we decided to implement the tracking of object references with the help of refinement runs.

Outbound object pointers - or in other words the set of objects a specific object references - can be determined with the help of  \inlinecode{Object>>outboundPointers}.
However, this method returns only the set of directly referenced objects.
For example, if an instance of \inlinecode{Subject} administers a collection of observers, it would return only the collection object itself, but not the observer objects contained in this collection.
Undoubtedly, this is the technically correct behavior.
But for our purposes it seems more suitable to return those indirectly referenced objects along with the actual references. 
More often than not, this is the actually relevant information for a developer.
Therefore, we decided to flatten the objects returned by \inlinecode{outboundPointers}, and generate identifiers as depicted in Section \ref{ss:ImplementationTracingIdentification} for the resulting objects.
Thus, objects that are wrapped in collections are included when a developer queries the outbound references of a specific object.

As pointed out in Section \ref{ss:ImplementationTracingIdentification}, objects will always be associated with the same identifier in repeated test executions.
This fact is exploited to map the references that are returned from refinement runs to the objects of the current trace.
However, to guarantee that the identifiers from the refinement runs correspond to the identifiers from the shallow analysis run, both executions have to be identical in terms of the tracing code that is executed.
Hence, it is not sufficient to instrument only the method that represents the current step of the execution history.
It is important that the method wrappers request identifiers from the tracer in the exact same order and quantity as in the shallow analysis run.
For instance, if an object identifier would not be requested, all subsequent assigned identifiers would differ from the ones assigned during the first tracing run.
Therefore, the refinement runs for object reference tracing mostly correspond to a shallow analysis run.

\section[Object Placement and Edge Routing]{Object Placement and Edge Routing%
\sectionmark{Graph Drawing}}
\sectionmark{Graph Drawing}
\label{s:ImplementationLayouting}
Object-oriented execution traces can be mapped to directed multigraphs.
Our implementation of \textsc{PathObjects} exploits this circumstance by using the state of the art in graph drawing to lay out objects and to route messages.
Thereby, two graph drawing frameworks are supported, namely \textsc{Roassal}\footnote{http://objectprofile.com/\#/pages/products/roassal/overview.html, last checked \today} and \textsc{Graphviz}\footnote{http://www.graphviz.org/, last checked \today}.
\textsc{Roassal} provides a Squeak-native implementation without external dependencies.
However, its results are deficient in terms of the aesthetic criteria we defined in Section \ref{s:ApproachLayout}.
For that reason, it only serves as a fallback solution.
The primary choice is \textsc{Graphviz}, which is integrated into \textsc{PathObjects} through the Squeak Foreign Function Interface.
The selection of the appropriate graph drawing framework is transparent to the user, and  \textsc{Graphviz} is selected automatically, provided that it is available on the host system.

\subsection{Mapping of Execution Traces to Graphs}
The structure of a \textsc{PathObjects} trace can be interpreted as directed multigraph $G$, meaning that the edges constitute a multiset rather than a set (cf. Definition \ref{eq:ImplementationGraph}).

\begin{equation}
G = (V, E)
\label{eq:ImplementationGraph}
\end{equation}

Objects are the set of vertices $V$ in this graph.
Message sends between two objects form the multiset of ordered pairs $E$, with each element having the form $(S,R)$.
Thereby, $S$ denotes the sender of a message, and $R$ its receiver.
Contrarily to the traditional definition of simple directed graphs \cite{berge_graphs_1985}, loops have to be allowed, since message exchange between objects is bidirectional more than often.

The multiset $M \subseteq E$ of messages that are exchanged between two specific objects $n_1$ and $n_2$ can be expressed with the help of formula \ref{eq:ImplementationGraphMessages}.

\begin{equation}
M(n_1, n_2) = \left\{ (s,r) \in E \phantom{i}| \phantom{i} (n_1 = s \land n_2 = r) \lor (n_1 = r \land n_2 = s) \right\}
\label{eq:ImplementationGraphMessages}
\end{equation}

Consequently, this allows to express the number of messages $C_M$ that are exchanged between two objects as cardinality of $M$ (cf. Formula \ref{eq:ImplementationGraphMessageCardinality}).

\begin{equation}
C_M(n_1, n_2) = \left\vert \phantom{i} M(n_1, n_2) \phantom{i} \right\vert
\label{eq:ImplementationGraphMessageCardinality}
\end{equation}

The transformation of a \textsc{PathObjects} trace to a graph structure is straightforward.
All occurring objects are mapped one-to-one to vertices of the graph.
However, not all messages have to be represented by corresponding edges due to the fact that only a predefined number of messages $M_{max}$ are maximally shown in a diagram at the same time.
For instance, if $M_{max}$ is set to $4$, the graph that is presented to the user can never show more than $4$ edges between two objects at once, even if $5$ or more messages are exchanged between them.
Consequently, the number of edges $C_E$ that are required between two specific vertices is the minimum of the predefined maximum value $M_{max}$ and the number of messages that are actually exchanged between them.

\begin{equation}
C_E(n_1, n_2) = min(C_M(n_1, n_2),\phantom{i} M_{max})
\end{equation}

\subsection{Graphviz Interoperability}

The graph drawing process consists of three phases.
First, a so-called dot-file is generated that contains a description of the graph in the format \textsc{Graphviz} expects.
Second, \textsc{Graphviz} is executed and the generated file is used as input parameter.
The result of this execution is a file that contains the coordinates of nodes as well as the coordinates of all edge bends.
Third, this file is parsed, the coordinates are translated to the Squeak coordinate system, and the morphs of the visualization are placed accordingly.

For each object that occurs in a trace, a node definition is added to the \textsc{Graphviz} graph definition.
For each pair of objects $(n_1, n_2)$, $C_E(n_1, n_2)$ edge definitions are added to the definition.

\begin{graphviz}[caption={Exemplary definition of a \textsc{Graphviz} graph, consisting of two nodes and three directed edges.}, label=lst:ImplementationGraphDot]
digraph Observers {
	graph [rankdir=LR, splines=ortho]
	node [shape=none, fixedsize=true]
	edge [arrowhead=none];
	
	1002 [width=1.54, height=0.38, label="a ClockTimer"];
	1027 [width=1.61, height=0.38, label="a AnalogClock"];
	...
	1002 -> 1002;
	1002 -> 1027;
	1002 -> 1027;
	...
}
\end{graphviz}

An example dot-file that results from the generation phase is given in Listing \ref{lst:ImplementationGraphDot}.
Line one starts with a definition of a directed graph.
Edge routing is configured to be orthogonal, and nodes and edges are configured to fit exactly to the extent definitions that are given later.
Lines six and seven show two exemplary definitions of nodes.
They start with an identifier, which conforms to the identifiers that are used during tracing, and is followed by a specification of the extents.
The labels are for illustration purposes only and are not required for the graph drawing functionality in our use case.
Lines $9-11$ show three definitions of edges, whereby one is a self loop and the other two illustrate that graphs are multigraphs.

Once the dot-file is written, a native installation of \textsc{Graphviz} is invoked through the Squeak Foreign Function Interface with the following command line:
\begin{center}
\inlinecode{dot graph.dot -s100 -y -Tplain-ext -o layout.dot}
\end{center}
The \inlinecode{-s100} argument instructs \textsc{Graphviz} to use a scale of 100 dots per inch.
This is necessary since \textsc{Graphviz} uses inches as internal measurement unit, and all coordinates and extent attributes have to be converted from points to inches and vice versa.
Using a multiple of 10 ensures that no rounding errors occur on that account.
The \inlinecode{-y} flag causes an inversion of the vertical axis, and thus the usage of a coordinate system with top-left origin.
Since Squeak uses such a coordinate system, this obviates the need to manually perform this conversion later.
Finally, \textsc{Graphviz} is instructed to use the \inlinecode{plain-ext} format for the results, and to write the calculated layout to a new file called \inlinecode{layout.dot}.

As soon as the \textsc{Graphviz} process exits, the resulting layout can be parsed from the generated file.
An example of this \inlinecode{plain-ext} format is given in Listing \ref{lst:ImplementationGraphDot}.
Each line represents either a graph definition, a node definition, an edge definition, or the stop identifier which marks the end of the file.
Analogous to the input values, the measurement unit for widths, heights, and coordinates is inches.
To convert those values to pixel values, they have to be multiplied with the resolution argument passed to the \inlinecode{dot} executable.

\begin{graphviz}[caption={Output in plain-ext format as produced by \textsc{Graphviz} for the input from Listing \ref{lst:ImplementationGraphDot}}, label=lst:ImplementationGraphPlainExt]
graph 1 14.306 4.5451
node 1002 9.9444 3.1528 1.5417 0.375 "a ClockTimer" ...
node 1027 13.042 2.0139 1.6111 0.375 "a DigitalClock" ...
...
edge 1002 1002 16 9.1735 3.255 8.6827 3.255 8.1389 3.255 ...
edge 1002 1027 7 9.8472 2.9659 9.8472 2.6371 9.8472 1.9896 ...
edge 1002 1027 7 10.719 3.2316 11.558 3.2316 12.778 3.2316 ...
...
stop
\end{graphviz}

The graph definition consists of the scaling factor, the width of the graph (1430px), and its height (454px).
The node definition contains the identifier of the node, the coordinates of its center, its extent, and its label.
The styling attributes like border- and background-colors were omitted from the output since they are not relevant for our use case.
An edge definition starts with the identifiers of the two nodes connected through this edge.
It is followed by the number of control points defining the B-spline forming the edge, which conform to edge bends in the case of orthogonal edge routing.
For instance, the edge in line 5 has 16 bends, and their x and y coordinates are subsequently enumerated in pairs.
Again, styling information like edge style and color have been omitted.

Once parsing is completed, the obtained layout can be used to adjust the extent of the \textsc{PatchObjects} canvas, and to place the objects on this canvas.
To route the edges for a specific step of the execution history, messages $m_i$ between two objects $n_1$ and $n_2$ are mapped to their representative $E_i$ from the multiset of edge definitions according to their relative position in the message sequence (cf. Definition \ref{eq:ImplementationGraphMapping}).
\begin{equation}
E_i(n_1, n_2, m_i) = m_i \bmod C_E(n_1, n_2)
\label{eq:ImplementationGraphMapping}
\end{equation}
This ensures that one message will always be mapped to the same edge when a user navigates to a specific point of the execution history repeatedly.
Furthermore, as pointed out above, it is guaranteed that enough edge definitions exists to depict any given point of the execution history.