\tolerance 1414
\hbadness 1414
\emergencystretch 1.5em
\hfuzz 0.3pt
\widowpenalty=10000
\vfuzz \hfuzz
\raggedbottom

%% highlighting
\makeatletter
\newenvironment{btHighlight}[1][]
{\begingroup\tikzset{bt@Highlight@par/.style={#1}}\begin{lrbox}{\@tempboxa}}
{\end{lrbox}\bt@HL@box[bt@Highlight@par]{\@tempboxa}\endgroup}

\definecolor{btBack1}{RGB}{209,229,254}
\definecolor{btBack2}{RGB}{164,204,254}
\definecolor{btBack3}{RGB}{81,158,251}
% \definecolor{btBack1}{RGB}{235,242,255}
% \definecolor{btBack2}{RGB}{215,222,255}
% \definecolor{btBack3}{RGB}{175,182,255}
%% \definecolor{btBack2}{RGB}{188,215,228}
%% \definecolor{btBack3}{RGB}{144,195,220}
\definecolor{btFont}{RGB}{66,128,204}
% \definecolor{btFont}{RGB}{95,99,179}
%% \definecolor{btFont}{RGB}{53,93,108}

\newcolumntype{g}{>{\columncolor{btBack2}}c}
\newcolumntype{R}[1]{>{\raggedleft\let\newline\\\arraybackslash\hspace{0pt}}b{#1}}

\newcommand\btHL[1][]{%
  \begin{btHighlight}[#1]\bgroup\aftergroup\bt@HL@endenv%
}
\def\bt@HL@endenv{%
  \end{btHighlight}%
  \egroup
}
\newcommand{\bt@HL@box}[2][]{%
  \tikz[#1]{%
    \pgfpathrectangle{\pgfpoint{1pt}{0pt}}{\pgfpoint{\wd #2}{\ht #2}}%
    \pgfusepath{use as bounding box}%
    \node[anchor=base west, fill=btBack1, outer sep=0pt,inner xsep=1pt, inner ysep=0pt, rounded corners=2pt, minimum height=\ht\strutbox+1pt,#1]{\raisebox{1pt}{\strut}\strut\usebox{#2}};
  }%
}
\makeatother

\newcommand{\cursor}{\textcolor{black}{\rule[-0.6ex]{1.1pt}{2.5ex}}}

\lstdefinelanguage[80]{Smalltalk}{
  basicstyle=\ttfamily\small,
  morekeywords={true,false,self,super,nil,thisContext},
  sensitive=true,
  morecomment=[s]{"}{"},
  morestring=[d]',
  style=SmalltalkStyle,
  numbers=left,
  basewidth={0.57em,0.45em},
  aboveskip=\bigskipamount,
  lineskip=.2ex,
  frame=single,
  framexleftmargin=5.0ex,
  xleftmargin=5ex,
  tabsize=4,
  morecomment=[s]{"}{"},
  commentstyle=\color{gray}\slshape,
  stringstyle=\color{-red!75!green!50!blue},
  captionpos=b
}
\lstdefinestyle{SmalltalkStyle}{
  literate={:=}{{\(\gets\)}}1
        {^}{{\(\uparrow\)}}1
}

\definecolor{mygreen}{rgb}{0,0.6,0}
\lstdefinelanguage{Graphviz}{
  basicstyle=\ttfamily\small,
  morekeywords={digraph, graph, node, edge, stop},
  sensitive=true,
  morestring=[d]",
  numbers=left,
  basewidth={0.57em,0.45em},
  aboveskip=\bigskipamount,
  lineskip=.2ex,
  frame=single,
  stringstyle=\color{mygreen},
  showspaces=false,
  showstringspaces=false,
  framexleftmargin=5.0ex,
  xleftmargin=5ex,
  tabsize=4,
  commentstyle=\color{gray}\slshape,
  captionpos=b
}

\lstset{defaultdialect=[80]{Smalltalk}}

% inline ocde
\definecolor{inlinecodeborder}{HTML}{DDDDDD}
\definecolor{inlinecodetext}{HTML}{666666}

\newcommand\inlinecode[2][]{
    \tikz[baseline=(s.base)]{
        \node(s)[
            rounded corners=2pt,
            %fill=blue!5,        % background color
            draw=inlinecodeborder, % border of box
            text=inlinecodetext, % text color
            inner xsep =3pt,    % horizontal space between text and border
            inner ysep =0pt,    % vertical space between text and border
            text height=1.8ex,    % height of box
            text depth =0.6ex,    % depth of box
            #1                  % other options
        ]{\texttt{#2}};
    }
}

% to indicate removed code
\newcommand{\coderem}{\textit{\footnotesize{[\ldots]}}}

% in default mode: use short inlines and flexible columns
\newcommand{\xpfDefaultMode}{
%\lstMakeShortInline[language=javascript]{^}
\lstset{columns=fullflexible}
}

% in listing mode: disable short inlines and use fixed columns 
\newcommand{\xpfListingMode}{
%\lstDeleteShortInline{^}
\lstset{columns=[c]fixed}
}
\xpfDefaultMode

\lstnewenvironment{smalltalk}[1][]{
	\lstset{language=Smalltalk,float=tb,#1}
	\xpfListingMode
}{
	\xpfDefaultMode
}

\lstnewenvironment{graphviz}[1][]{
	\lstset{language=Graphviz,float=tb,#1}
	\xpfListingMode
}{
	\xpfDefaultMode
}

\selectlanguage{english}

\floatplacement{figure}{htbp}

% special listing style for listings and subfloat listings
\makeatletter
\renewcommand*{\listof}[2]{%
  \@ifundefined{ext@#1}{\float@error{#1}}{%
    \expandafter\let\csname l@#1\endcsname \l@figure
    \float@listhead{#2}%
    \begingroup
      \@starttoc{\@nameuse{ext@#1}}%
    \endgroup}}
\makeatother

%\newfloat{mylisting}{htbp}{lol}[chapter]
%\floatname{mylisting}{Listing}
%\newsubfloat{mylisting}
%\renewcommand{\lstlistoflistings}{\listof{mylisting}{List of Listings}}
%\providecommand*\theHmylisting{\themylisting}
%\let\Chapter\chapter
%\def\chapter{\addtocontents{lol}{\protect\addvspace{10pt}}\Chapter}

\renewcommand{\lstlistlistingname}{List of Listings}
\newcommand{\eg}{\mbox{e.g.}\xspace}
\newcommand{\Eg}{\mbox{E.g.}\xspace}
\newcommand{\Ie}{\mbox{I.e.}\xspace}
\newcommand{\ie}{\mbox{i.e.}\xspace}
\newcommand{\Th}{\textsuperscript{th}\xspace}

\renewcommand{\mkcitation}[1]{\par\hfill#1}
\renewcommand{\mkblockquote}[4]{``#1''#2#4#3}

\makeatletter
\renewcommand{\@makechapterhead}[1]{%
\vspace*{0 pt}%
{\setlength{\parindent}{0pt} \raggedright \normalfont
\bfseries\LARGE
\ifnum \value{secnumdepth}>1 
   \if@mainmatter\thechapter\hskip 15\p@\fi%
\fi
#1\par\nobreak\vspace{15 pt}}}
\makeatother

\makeatletter
\renewcommand{\@chapapp}{}
\newenvironment{chapquote}[2][2em]
  {\setlength{\@tempdima}{#1}%
   \def\chapquote@author{#2}%
   \parshape 1 \@tempdima \dimexpr\textwidth-2\@tempdima\relax%
   \itshape}
  {\par\normalfont\hfill--\ \chapquote@author\hspace*{\@tempdima}\par\bigskip}
\makeatother

%\renewcommand{\thechapter}{$\arabic{chapter}$}
%\renewcommand{\thesection}{$\arabic{section}$}
%\renewcommand{\thesubsection}{$\arabic{subsection}$}

%\glossarystyle{index}

\newcommand{\figscale}{0.65}
\frenchspacing
\linespread{1.05}

\setlength{\ULdepth}{0.4ex}

\tikzstyle{tag} = [circle, inner xsep=0pt, fill=lightgray!50!gray, text centered]
%\newcommand{\tag}[1]{\tikz[baseline=(t.base)] \node[tag](t){$#1$};}

\newcommand{\cc}[1]{{\fontsize{10}{11}\selectfont \texttt{#1}}}
\newcommand{\ccb}[1]{{\fontsize{10}{11}\selectfont \texttt{\textbf{#1}}}}

\setlength{\parindent}{0pt}
\setlength{\parskip}{1ex plus 0.5ex minus 0.2ex}