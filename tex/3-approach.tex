\chapter{Revealing Object Interactions with \textsc{PathObjects}}
\chaptermark{Revealing Object Interactions}
\label{c:approach}

\section{Concept}

\section{Diagram Notation}
\subsection{Basic Concept}
\subsection{Diagram Aesthetics}
The object communication diagrams of \textsc{PathObjects} form graph structures, whereby  objects constitute nodes, and the arrows representing messages constitute edges.
Graph drawing is a research area with a long history of its own, and the aesthetic properties that should be aimed at in order to maximize perceivability are well known.
Consequently, it stands to reason that these criteria should also be applied to the diagrams created through our approach.

\paragraph{Crossing Minimization} User studies suggest that the most important aesthetic criterion is the minimization of edge crossings \cite{purchase_effective_2000, purchase_graph_2004, purchase_graph_2010}.
The more edges cross each other, the harder it becomes for the human eye to keep track of which nodes are connected by an edge.
Graph drawings that do not entail edge crossings are called planar, and are regarded as the best-case scenario.
But although planar graph drawing algorithms are comparatively simple, the excessive optimization of drawings in these premises will most likely interfere with other aesthetic criteria.

\paragraph{Bend Minimization}

\paragraph{Area Minimization}

\paragraph{Symmetry}

\paragraph{Clustering}

\paragraph{Orthogonality}

\subsection{Automatic Diagram Layout}
In general, there are many profound arguments for generating diagram layouts automatically.
For instance, automated drawing of diagrams has the side-effect that conformance to a certain style guide .
Furthermore, the costs of communication and thus the total costs of production and maintenance can be reduced by use of automated drawing of diagrams.
But the strongest reason to rely on the automated generation of diagram layouts becomes evident when taking the intended use of \textsc{PathTools} into account.


\section{Tools}
\subsection{Navigation}
\subsection{Exploration}
\subsection{Focusing}
\subsection{Information Layers}