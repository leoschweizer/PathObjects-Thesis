\chapter*{Abstract\markboth{Abstract}{Abstract}}
\label{abstract}

\begin{minipage}[t]{\textwidth}

	\setlength{\parindent}{0pt}
	\setlength{\parskip}{1ex plus 0.5ex minus 0.2ex}

Program comprehension is a tedious necessity during software development with complex cognitive demands.
Reportedly, it amounts to up to 60\% of the software engineering time and labor.
One reason that makes program comprehension such a time-consuming activity is that it requires developers to bridge a gap of abstraction levels.
Namely, the low-level abstractions of the inner workings of computers and the high-level abstractions of object-oriented programming languages and design documents diverge.
Unfortunately, widespread development tools contribute little to bridge this gap, and present information about program behavior with technical views instead of adapting to the abstractions of object-orientation.
However, to close this gap, an information presentation at the abstraction level of objects and their interactions would be desirable, which in turn could contribute to reduce the time required for program comprehension.

Therefore, this thesis presents \textsc{PathObjects}, an interactive way of diagraming program behavior at the abstraction level of object interactions.
It allows developers to navigate and explore execution traces through an immediate information presentation while maintaining a low memory footprint.
We put this concept into practice with an implementation for the Squeak/Smalltalk environment.
With the aid of a user study, we show that \textsc{PathObjects} can help developers to comprehend an unknown software system faster and to a higher degree compared to the standard development tools.
We conclude that our concept can assist developers in object-oriented program comprehension by closing the gap of abstraction levels, and thus represents a step towards the reduction of the required time of this laborious activity.

\end{minipage}

%% \cleardoublepage

\selectlanguage{ngerman}

\chapter*{Zusammenfassung\markboth{Zusammenfassung}{Zusammenfassung}}

\begin{minipage}[t]{\textwidth}

	\setlength{\parindent}{0pt}
	\setlength{\parskip}{1ex plus 0.5ex minus 0.2ex}

\begin{flushleft}
{\Large \textsc{PathObjects} --- Offenlegung von Objektinteraktionen zur Unterstützung von Entwicklern beim Verstehen von Programmen}
\end{flushleft}

Das Verstehen von Programmen ist eine mühsame Notwendigkeit während der Softwareentwicklung und stellt hohe kognitive Ansprüche.
Dementsprechend wird berichtet, dass Entwickler bis zu 60\% ihrer Arbeitszeit damit verbringen.
Ein Grund, der das Verstehen von Programmen zu einer derart zeitintensiven Tätigkeit macht, ist, dass Entwickler dabei eine Lücke zwischen verschiedenen Abstraktionsebenen überbrücken müssen.
Diese Divergenz offenbart sich zwischen dem niedrigen Abstraktionsgrad der internen Funktionsweise von Computern einerseits und dem hohen Abstraktionsgrad objektorientierter Programmiersprachen und Entwurfsdokumente andererseits.
Leider tragen weitverbreitete Entwicklungswerkzeuge wenig dazu bei, diese Lücke zu schließen, und stellen Informationen über das Verhalten von Programmen mit technischen Ansichten dar, anstatt diese an die Abstraktionen der Objektorientierung anzupassen.
Eine Informationsdarstellung auf der Abstraktionsebene von Objekten und ihren Interaktionen wäre wünschenswert, um dieser Diskrepanz entgegenzuwirken, und um somit die Zeit, die für das Verstehen von Programmen benötigt wird, zu reduzieren.

Deshalb stellen wir in dieser Arbeit \textsc{PathObjects} vor, ein interaktives Verfahren zur Darstellung von Programmverhalten auf der Abstraktionsebene von Objektinteraktionen.
Es ermöglicht, Ausführungstraces mit geringer Wartezeit und niedrigen Speicheranforderungen nachzuvollziehen und zu untersuchen.
Mittels einer Implementierung für die Squeak/Smalltalk-Umgebung setzen wir dieses Konzept in die Praxis um.
Anhand einer Nutzerstudie zeigen wir außerdem, dass \textsc{PathObjects} Entwicklern helfen kann, ihnen unbekannte Softwaresysteme schneller und umfassender zu verstehen, als dies im Vergleich zu den Standardentwicklungswerkzeugen der Fall ist.
Wir schlussfolgern daraus, dass unser Konzept Entwickler beim Verständnis objektorientierter Programme unterstützen kann, indem es die Lücke zwischen den Abstraktionsebenen schließt, und dass es somit dazu beitragen kann, die benötigte Zeit für diese aufwändige Tätigkeit zu reduzieren.

\end{minipage}

\selectlanguage{english}
