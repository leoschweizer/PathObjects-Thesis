\chapter*{Abstract\markboth{Abstract}{Abstract}}
\label{abstract}

\begin{minipage}[t]{\textwidth}

	\setlength{\parindent}{0pt}
	\setlength{\parskip}{1ex plus 0.5ex minus 0.2ex}

Program comprehension is a tedious necessity during software development with complex cognitive demands.
Reportedly, it amounts to up to 60\% of the software engineering time and labor.
One reason that makes program comprehension such a time-consuming activity is that it requires developers to bridge a gap of abstraction levels.
Namely, the low-level abstractions of the inner workings of computers and the high-level abstractions of object-oriented programming languages and design documents diverge.
Unfortunately, widespread development tools contribute little to bridge this gap, and present information about program behavior with technical views instead of adapting to the abstractions of the respective programming language.
However, to close this gap, an information presentation at the abstraction level of objects and their interactions would be desirable, which in turn could contribute to reduce the time share of program comprehension.

Therefore, this thesis presents \textsc{PathObjects}, an interactive way of diagraming program behavior at the abstraction level of object interactions.
It allows developers to navigate and explore execution traces through an object-oriented information presentation while enabling immediacy and maintaining a low memory footprint.
We put this concept into practice with an implementation for the Squeak/Smalltalk environment.
With the aid of a user study, we show that \textsc{PathObjects} can help developers to comprehend an unknown software system faster and to a higher degree compared to the standard development tools.
We conclude that our concept can assist developers in object-oriented program comprehension by closing the gap of abstraction levels, and thus represents a step towards the reduction of the time share of this laborious activity.

\end{minipage}

%% \cleardoublepage

\selectlanguage{ngerman}

\chapter*{Zusammenfassung\markboth{Zusammenfassung}{Zusammenfassung}}

\begin{minipage}[t]{\textwidth}

	\setlength{\parindent}{0pt}
	\setlength{\parskip}{1ex plus 0.5ex minus 0.2ex}

\begin{flushleft}
{\Large \textsc{PathObjects} --- Offenlegung von Objektinteraktionen zur Unterstützung von Entwicklern beim Verständnis von Programmen}
\end{flushleft}

\todo[inline]{...}

\end{minipage}

\selectlanguage{english}
