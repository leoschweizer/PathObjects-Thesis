\chapter*{Abstract\markboth{Abstract}{Abstract}}
\label{abstract}

\begin{minipage}[t]{\textwidth}

	\setlength{\parindent}{0pt}
	\setlength{\parskip}{1ex plus 0.5ex minus 0.2ex}

Program comprehension is a tedious necessity with complex cognitive demands and reportedly amounts to up to 60\% of the software engineering effort.
One reason that makes program comprehension such a time-consuming activity is that it requires developers to bridge a gap of abstraction levels.
Namely, the low-level abstractions of the inner workings of computers and the high-level abstractions of object-oriented programming languages and design documents diverge widely.
Surprisingly, widespread development tools contribute little to bridge this gap, and present information about program behavior with technical views.
However, it stands to reason that an information presentation at the abstraction level of objects and their interactions can contribute to close this gap of abstraction levels, and consequently can help to reduce the time share of program comprehension.

Therefore, this thesis presents \textsc{PathObjects}, an interactive way of diagraming object interactions that allows developers to navigate and explore object-oriented program behavior while enabling immediacy and maintaining a low memory footprint.
We validate this concept with the help of an implementation for Squeak/Smalltalk.
With the aid of a user study, we show that \textsc{PathObjects} can help developers to comprehend an unknown software system faster and to a higher degree in comparison to the Squeak standard development tools.
We conclude that our concept can assist developers in object-oriented program comprehension by closing the gap of abstraction levels, and thus represents a step towards the reduction of the time share of this activity.

\end{minipage}

%% \cleardoublepage

\selectlanguage{ngerman}

\chapter*{Zusammenfassung\markboth{Zusammenfassung}{Zusammenfassung}}

\begin{minipage}[t]{\textwidth}

	\setlength{\parindent}{0pt}
	\setlength{\parskip}{1ex plus 0.5ex minus 0.2ex}

\begin{flushleft}
{\Large \textsc{PathObjects} --- Offenlegung von Objektinteraktionen zur Unterstützung von Entwicklern beim Verständnis von Programmen}
\end{flushleft}

\todo[inline]{...}

\end{minipage}

\selectlanguage{english}
