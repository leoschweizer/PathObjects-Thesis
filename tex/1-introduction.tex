\chapter{Introduction}
\label{c:introduction}

Program comprehension is an essential requirement during software development.
During this process, developers construct a mental model that maps the features of a software system to their source code representation \cite{latoza_maintaining_2006}.
But although development tools have evolved notably, it still remains a tedious task.
Consequently, it is reported that developers spend up to 60\% of the software engineering effort on building a satisfactory understanding of the system at hand \cite{corbi_program_1989, basili_evolving_1997, cornelissen_execution_2008}.

At the same time, "programmers tend to adapt to the level of representation
provided by the computer, instead of adapting the computers representations
to their perceptive abilities" \cite{diehl_software_2007}.
As a result, there is a huge gap between the metaphors and concepts used during development, and the representation tools provide of a program's behavior.

Conventional symbolic debuggers that commonly are included in object-oriented development environments constitute a paragon for this phenomenon.

\todo[inline]{Context}
\todo[inline]{Problem}
\todo[inline]{Significance -> Research Question}
\todo[inline]{Solution}

\section{Contributions}
\label{s:contributions}

\begin{description}[leftmargin=0pt]
\item[First Contribution]
...
\end{description}



\section{Thesis Structure}
\label{s:structure}

The remainder of this thesis is structured as follows:
...

%% Sub chapters in separate files:
% \input{1-1-contributions}
% \input{1-2-structure}