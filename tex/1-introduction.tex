\chapter{Introduction}
\label{c:introduction}

Program comprehension is an essential requirement during software development.
During this process, developers construct a mental model that maps the features of a software system to their source code representation \cite{latoza_maintaining_2006}.
But while the programming languages and concepts developers use have evolved notably from eighty-column cards and assembler to object-oriented programming, the further development of established tools that are supposed to help in the understanding of software systems seemingly was not able to keep pace.

In other words, Diehl states that "programmers tend to adapt to the level of representation provided by the computer, instead of adapting the computer's representations to their perceptive abilities" \cite{diehl_software_2007}.
As a result, there is a huge gap between the metaphors and concepts used during development, and the representation of program behavior tools provide.

Conventional symbolic debuggers that commonly are included in object-oriented development environments constitute a paragon for this phenomenon.
While developers design a system with objects and their interactions in mind, the information presentation of those debuggers is centered around the call stack of the current execution.
This presentation rather conforms to the processes on machine level than to the object-oriented concepts...
It is not particularly suited to answer questions that often arise during development.

Consequently, it is reported that developers spend up to 60\% of the software engineering effort on building a satisfactory understanding of the system at hand \cite{corbi_program_1989, basili_evolving_1997, ducasse_class_2005, rothlisberger_feature_2007, cornelissen_execution_2008}.


\todo[inline]{Context}
\todo[inline]{Problem}
\todo[inline]{Significance -> Research Question}
\begin{quote}
Can the behavior of software systems be presented at the abstraction level of objects while maintaining immediacy and perceiveability?
\end{quote}
\todo[inline]{Solution}

\section{Contributions}
\label{s:contributions}

\begin{description}[leftmargin=0pt]
\item[First Contribution]
...
\end{description}



\section{Thesis Structure}
\label{s:structure}

The remainder of this thesis is structured as follows:
...

%% Sub chapters in separate files:
% \input{1-1-contributions}
% \input{1-2-structure}