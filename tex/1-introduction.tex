\chapter{Introduction}
\label{c:introduction}
Program comprehension is an integral part of the software development process.
For instance, maintenance tasks such as bug fixing, feature enhancement, refactoring, or documentation cannot be performed without having a sufficient understanding of the system at hand.
However, program comprehension also is a time-consuming activity with complex cognitive demands, and developers reportedly spend up to 60\% of the software engineering effort on it \cite{corbi_program_1989, basili_evolving_1997, ducasse_class_2005, rothlisberger_feature_2007, cornelissen_execution_2008}.
As a result, software maintainance is expensive, and is reported to account for 40\%-80\% of the total costs of software development \cite{glass_frequently_2001}.
Consequently, it stands to reason that reducing the time share of program comprehension could contribute to a significant reduction of the overall costs of software engineering.

This raises the question why program comprehension is such a tedious task.
During this activity, developers construct a mental model that maps the behavior of a software system to its source code representation \cite{latoza_maintaining_2006}.
In doing so, numerous gaps between different conceptual areas have to be bridged \cite{kent_program_1996}.
First, developers have to map the concepts of the application domain to the abstract concepts that are found in the source code.
Second, models and software artifacts often are not kept in a coherent state, which forces developers to understand a software system whose documentation can be out of date.
Third, there is a gap between the formal world of computer programs with strict syntactic and semantic rules, and the nature of human cognition which has to identify patterns in the source code and build correct relations between them associatively.
Fourth, developers have to synchronize bottom-up and top-down activities, namely program analysis and model synthesis.

But most importantly in the context of this thesis, there is a fifth gap between the low-level functional principals of physical machines and the high-level abstractions of programming languages and design documents.
On the one hand, a program is nothing more than a number of CPU register values when executed by a computer.
On the other hand, the same program may have been designed with high-level abstractions like object-orientation in mind.
As a consequence, a further challenge for developers during program comprehension is the selection of relevant details from the low-level representations provided by computers, and the building of adequate abstractions on the basis of these details.

Surprisingly, wide-spread development tools contribute little to close this gap.
Conventional symbolic debuggers that commonly are included in object-oriented development environments stand exemplary for this phenomenon.
The information presentation of those debuggers is centered around the call stack of the current execution.
This presentation rather conforms to the processes on machine level than to the object-oriented concepts used during the development of the system.
Consequently, it is not particularly suited to answer questions about object-oriented programs such as which objects are involved in the current computation, and which other objects they are communicating with.
The mere fact that these debuggers could just as well be used for the debugging of procedural programs illustrates how little they are adapted to the domain of object-orientation.

\section[Challenges in Object-Oriented Program Comprehension]{Challenges in Object-Oriented Program Comprehension%
\sectionmark{Challenges}}
\sectionmark{Challenges}
The way program behavior is presented to developers is characterized by the low level concepts computers operate on, instead of the high level concepts that are used to design software systems.
The result of this phenomenon is a huge gap between the metaphors and concepts used during development, and the representation of program behavior tools provide.
Unsurprisingly, research results in the domain of object-oriented program comprehension suggest that objects and their interactions are an essential part of mental models that are built during this activity \cite{burkhardt_mental_1997}.
At the same time, the way in which information is presented plays an important role in the process of human reasoning and thus in the construction of mental models \cite{diehl_software_2007}.
Consequently, it seems reasonable that a reduction of the gap between the information presentation of a program's behavior and the concepts that are used at design time would help to reduce the amount of time developers have to spend on program comprehension.
If an information presentation at the abstraction level of objects and their interactions could be provided, developers no longer would have to map the low-level concepts provided by development tools to the high-level concepts used mentally.

This situation raises the question why such an information presentation has not been established so far.
The problem is that in order to visualize program behavior at the abstraction level of objects, the application of dynamic analysis techniques is inevitable.
This means that information about the behavior of a program has to be collected during actual executions of the program under investigation, because it is impossible to derive this kind of information from the source code alone.
The latter models the structure of a system in terms of classes and their relationships rather than its actual behavior.
But due to object-oriented concepts like polymorphism and late binding, the objects that are actually participating in a computation and the messages that are exchanged between them are only known at runtime.

However, dynamic analysis has four closely related problems that constrain many of its application areas.
First, to collect runtime information, the execution of a program has to be recorded.
Typically, this is done by instrumentation at the source code or byte code level or through the utilization of debug interfaces.
The side effect of such manipulations is that they often decelerate the execution speed of a program excessively.
The runtime costs of instrumentation are reported to range from around 10-15 times the costs of standalone executions up to 115 times in worst case scenarios \cite{pothier_scalable_2007, karran_synctrace:_2013}.
Consequently, developers have to wait a significant amount of time for tracing runs to finish, thereby increasing the amount of time required for program comprehension.

The second problem is the huge amount of data that arises even during short executions of software systems.
For instance, Karran et al. report that the start-up of the Firefox browser alone yields 1GB of trace data \cite{karran_extraction_2013}.
Pothier et al. state that a tracing session of the Eclipse IDE that involved the creation of classes, editing of source code, and a step-by-step execution with the debugger resulted in a trace of 33GB \cite{pothier_scalable_2007}.
This issue is so prevalent that the topic of trace size reduction has become a research area of its own.

As a result, the third problem turns out to be the time that is required to analyze and visualize traces.
Due to the huge amount of data, they often cannot be kept in main memory, which slows down the process further.
Again, time that developers have to wait for results is time that slows down the process of program comprehension and thus should preferably be spent on more productive tasks.

The fourth and last problem also is a direct result of overly large traces.
The visualizations of traces themselves grow proportional to the size of the traces.
This makes it increasingly difficult for developers to perceive and comprehend the actual information which is encoded in the visualization.

\emph{Step-wise run-time analysis} has been proposed as lightweight alternative to conventional tracing approaches to cope with some of the shortcomings of dynamic analysis  \cite{perscheid_immediacy_2010}.
The fundamental idea is to use test cases as reproducible entry points into the behavior of specific usage scenarios of a system.
Instead of collecting potentially useful information in its entirety up front, a shallow analysis is performed during the first tracing run that only yields the bare minimum of data required for the reconstruction of test executions.
During the inspection of such a shallow trace, developers can perform refinement runs to collect further information of interest, such as the internal state of objects or the specific values that are used as arguments.
As a result, the approach enables immediacy and displays only that information to  developers that is actually relevant.

However, this solution still lacks an appropriate information representation at the abstraction level of objects.
The primary purpose of the tool that implements step-wise run-time analysis is back-in-time debugging.
Therefore, it presents execution traces in the form of call trees.
Again, this presentation conforms to the technical procedures rather than being guided by the high-level concepts of object-orientation.
Therefore, the main research question that is tackled in this thesis is:

\begin{quote}
1) Can the behavior of software systems be presented at the abstraction level of objects while maintaining immediacy and perceivability?
\end{quote}

Since it is only an assumption that such an information presentation would actually help developers to understand the software system at hand better, the second research question we derive is:

\begin{quote}
2) Can such an information presentation actually aid in the process of program comprehension?
\end{quote}

Provided that an information presentation at the abstraction level of objects and their interactions actually can help to bridge the gap of abstraction levels developers face, it could play an important role in the reduction of the time that is spent on program comprehension.
Furthermore, since program comprehension is an essential element of software maintenance tasks, and since the latter make up for the majority of the total costs of software development,
such an information presentation could represent a step towards the reduction of those costs.


\section[Revealing Object Interactions with PathObjects]{Revealing Object Interactions with PathObjects%
\sectionmark{Revealing Object Interactions}}
\sectionmark{Revealing Object Interactions}
In this work, we tackle the two related problems implied by our first research question: first, traditional development tools tend to adapt the low level concepts of computers instead of adapting the high level concepts used during development.
Second, an information representation of objects and their interactions is hard to provide in a beneficial manner due to the scalability issues of dynamic analysis techniques.
Therefore, we extend the step-wise run-time analysis tracing technique in a way that preserves its immediate character and low memory footprint while at the same time making it suitable for the reconstruction of object interactions.
Based on this extension, we implement \textsc{PathObjects}, an interactive information presentation at the abstraction level of objects that aims at avoiding the information overload caused by existing dynamic analysis tools.
\textsc{PathObjects} diagrams reuse well-known concepts from the Unified Modeling Language and cover objects, message exchanges, state changes, and the roles objects play in communications.
On top of that, the \textsc{PathObjects} concept includes three groups of interactive diagram components that aim at easing perceivability of trace information, namely \emph{navigation}, \emph{exploration}, and \emph{focusing}.

The first component group covers navigational aspects.
To allow developers to navigate through the temporal dimension of execution traces, we provide the \emph{timeline} component.
Similarly, the intended use-case of the \emph{mini-map} component is the navigation through the spatial domain of execution traces.
These two components aim at an exploratory usage.
On top of that, we suggest the \emph{search bar} that facilitates target-oriented navigation, meaning that developers are enabled to find specific entities of interest provided that they are already known.

The second component group covers the explorative aspects of trace comprehension.
We present three interactive components in this context: \emph{state exploration}, \emph{reference exploration}, and \emph{source code exploration}.
State exploration enables developers to keep track of object state changes over the course of time.
Among other things, the reasons for program behavior thus can be revealed.
Reference exploration can be used to monitor the outbound object pointers of specific objects.
In this way, it can be revealed how and when objects retain and release references to other objects.
While these two components cover the exploration of the runtime state of a system, the fundamental goal of program comprehension is to map this behavior to its source code representation.
Consequently, source code exploration keeps connections to the underlying implementation of a system at all times, and allow developers to explore the context and the implementations of specific messages.

The third component group deals with the potential information overload of execution trace visualizations.
To allow developers to remove irrelevant information from \textsc{PathObjects} diagrams, we provide the \emph{object filter} and \emph{message filter} as interactive components.
With the aid of these components, developers are enabled to focus on relevant aspects of execution traces, and to construct terse scenarios that cover precisely the relevant amount of information.

Finally, we present information layers, a concept that allows developers to augment \textsc{PathObjects} diagrams with use-case specific information.
We show how information from various domains can be integrated into the visualization of object interactions, and how the need to manually collect information from a multitude of separate tools can be eliminated thereby.

In summary, the contributions of this thesis are:

\begin{itemize}
\renewcommand{\labelitemiv}{$\ast$}

\item \textbf{An Interactive Diagram Notation for Object Interactions} We present an interactive way of diagramming object states and interactions that allows users to navigate through program executions, to explore object states and relationships, to focus on specific parts of interest, and to augment the displayed information through user-defined information layers.

\item \textbf{A Lightweight Approach for Tracing Object Interactions} We present an extension of the step-wise run-time analysis approach that allows to reconstruct object interactions from traces while maintaining immediacy and a low memory footprint.

\item \textbf{A Prototypic Implementation for Squeak/Smalltalk} We validate the applicability of our approach with the help of an implementation for the Squeak development environment.
\end{itemize}

Furthermore, we use this prototype to verify that our proposed solution actually can help developers in the process of program comprehension with the help of a user study. \todo{summarize results}

\section{Thesis Structure}
The remainder of this thesis is structured as follows:
Chapter \ref{c:Background} looks into the background of program comprehension, shows why dynamic analysis is an inevitable tool for the comprehension of object-oriented programs, and presents established diagram notations that cover the state of objects and their interactions.
Thereby, the techniques and their inadequacy for our objectives are discussed.
Chapter \ref{c:approach} presents \textsc{PathObjects} diagrams as proposed solution to those problems, and describes the interactive components that facilitate trace navigation, exploration, focusing, and the augmentation with use-case specific information.
Chapter \ref{c:implementation} presents an implementation of \textsc{PathObjects} for the Squeak/Smalltalk environment and describes the required extensions of the step-wise run-time analysis tracing framework.
Chapter \ref{c:discussion} evaluates our approach with the help of a performance analysis and a user study. 
The chapter concludes with a discussion of the limitations of our approach and its applicability to other programming languages and environments.
Chapter \ref{c:relatedwork} shows related work in the areas of trace visualization, debugging, and education.
Chapter \ref{c:conclusions} concludes this work and presents reasonable directions for future research.