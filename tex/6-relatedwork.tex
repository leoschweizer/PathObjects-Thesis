\chapter{Related work}
\label{c:relatedwork}

Systä et al. present a reverse engineering environment called \textsc{Shimba} \cite{systa_shimba_2001}, which supports both static and dynamic analysis of Java software systems.
Static information about software entities like classes, interfaces and their relationships is extracted from the bytecode representation and visualized using \textsc{Rigi} \cite{muller_understanding_1993} in the form of directed dependency graphs.
Besides the computation of some of the metrics of Chidamber's and Kemerer's suite \cite{chidamber_metrics_1994}, \textsc{Rigi} also can be utilized to build abstractions, for instance by aggregating classes into their respective packages.
Runtime traces are collected with the help of a customized SDK debugger and can be synthesized as UML-like sequence and statechart diagrams through \textsc{SCED} \cite{koskimies_automated_1998,systa_understanding_2000}.
\textsc{Shimba} supports bidirectional model slicing between \textsc{Rigi} graphs and \textsc{SCED} diagrams.
On one side, \textsc{Rigi} can be used to restrict the amount of collected execution events through the selection of components of interest. Thus, the scope of the generated \textsc{SCED} diagrams is reduced.
On the other side, starting from a \textsc{SCED} diagram, static views covering the affected components can be obtained.