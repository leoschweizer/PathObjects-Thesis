\chapter{Conclusions}
\label{c:conclusions}
This chapter summarizes our work and shows potential areas for future research.
First, Section \ref{s:ConclusionsSummary} recapitulates the contributions of this thesis, and shows how they answer our research questions.
Second, Section \ref{s:ConclusionsFuture} concludes this thesis by presenting topic areas whose research could help to improve the concept of \textsc{PathObjects} even further. 

\section{Summary}
\label{s:ConclusionsSummary}
In this thesis, we tackle the gap of abstraction levels developers face during the comprehension of object-oriented programs.
While software systems are designed on high abstraction levels and implemented with programming languages that reflect these abstractions, the information presentation widespread development tools provide is oriented on the low level concepts of the inner workings of computers.
One of the reasons why information about program behavior is not presented at the abstraction level of its implementation and the corresponding design documents is that dynamic analysis is inevitable to gather the required information.
However, dynamic analysis suffers from four closely related scalability issues that have constrained many of its application areas for a long time.
Step-wise run-time analysis has been proposed as lightweight alternative to conventional dynamic analysis approaches and promises to solve most of these scalability issues.
It features a low memory footprint and makes immediate information presentations of program behavior possible.
Nevertheless, the back-in-time debugger called \textsc{PathFinder} that implements this approach still does not contribute to bridge the gap of abstraction levels, since the information presentation is centered around a call tree visualization.
Again, this presentation rather conforms to the technical procedures of the execution of a program than to the abstractions of object-orientation.

Consequently, the primary research question we drive is: \textit{Can the behavior of software systems be presented at the abstraction level of objects while maintaining immediacy and perceivability?}.
Since it is only an assumption that the bridging of the gap of abstraction levels through such an information presentation would actually assist developers to understand the software system at hand better, the second research question that is tackled in this thesis is:
\textit{Can such an information presentation actually aid in the process of program comprehension}?

As solution, we present \textsc{PathObjects}, an interactive diagraming concept for the presentation of object-oriented program behavior.
In contrast to widespread development tools, it focuses on an information presentation at the abstraction level of objects, and introduces a number of additional components that assist developers in the comprehension of execution traces.
First, \textit{navigational} components allow developers to browse the temporal and spatial domains of execution traces exploratory and to search for entities of interest in a target-oriented manner.
Second, \textit{explorative} components enable developers to explore the internal states of objects and their development over the course of time, to track references to other objects, and to explore the source code of the underlying implementation.
Third, \textit{focusing} components facilitate the creation of custom-tailored information presentations by permitting developers to define objects and messages of interest.
Beyond that, we present \textit{information layers} which allow developers to augment \textsc{PathObjects} diagrams with additional use-case specific information from various sources.
Thus, the diverse information requirements of different comprehension tasks can be met.

To validate the applicability of the concept, we put it into practice in the form of an implementation for the Squeak/Smalltalk environment.
This implementation uses an extended form of the step-wise run-time analysis approach as source of tracing information.
With the help of this implementation, we show that the majority of test cases of real-world software projects can presented to developers immediately.
Furthermore, we show that the additional tracing effort which is necessary for the reconstruction of object interactions does not invalidate the low memory footprint of the underlying tracing approach.

With regard to the second research question, we also used this implementation of \textsc{PathObjects} to conduct a user study.
This evaluation shows \todo{summarize results}
Consequently, we come to the conclusion that \textsc{PathObjects} actually can assist developers in object-oriented program comprehension, and thus can contribute to reduce the time that is spent on this activity.
Furthermore, since program comprehension is an essential requirement of software maintenance tasks which are accountable for the majority of the total costs of software development, we conclude that \textsc{PathObjects} constitutes a step towards the reduction of those costs.

\section{Outlook}
\label{s:ConclusionsFuture}
The evaluation of \textsc{PathObjects} led to new questions whose research could contribute to the further development of the concept.
First, the application of pattern recognition techniques could help to improve the perceivability of diagrams.
Second, advanced querying could allow developers to formulate very specific queries for entities of interest.
Third, it might be beneficial to support developers in finding and selecting suitable entry points.

\paragraph{Pattern Recognition} Many trace visualization approaches that cover complete program executions employ pattern recognition techniques to reduce the traces to a manageable size (cf. Chapter \ref{c:relatedwork}).
Our approach of using test cases as representative program executions in conjunction with Step-wise Run-time Analysis mostly solves the trace size scalability issue in terms of processing time and memory consumption.
Nevertheless, some test cases are still too large to serve as foundations of comprehensible scenarios.
However, our observations are that the traces of large test cases are often repetitive in nature. Commonly, the same set of operations is performed in the same order on different objects.
Recurring instances of such execution patterns could be omitted, presenting only relevant instances to the user, and thus avoiding cognitive overload.
In this regard, the execution structure of such test cases might be particularly advantageous for the application of pattern detection techniques.
The suitability of such an approach could be the subject of further research.

\paragraph{Advanced Querying} The keyword-based search engine in it's current form allows to answer questions like "when is an object used (next)?", "where is a specific method called?" or "where are instances of a specific class used?".
A more sophisticated search functionality with a dedicated query language - be it textual or graphical - would be desirable.
Three categories of questions such a query language could tackle are apparent.
The first category are queries over the message graph or call tree respectively.
For instance, those could include "when is a specific message sent to a specific object" or "which messages are follow-up messages of a specific one".
The second category consists of questions about the object or reference graph.
One could ask questions like "at which point in time does a specific object obtain or release a reference to another object?" or "which objects retain references on a specific object at a given point of time?".
The third and last category are questions about object state and state changes.
Ko and Myers have shown with \textsc{Whyline} that this area can very well be covered by a graphical query "editor" \cite{ko_debugging_2008}.
Such an approach could also be integrated into \textsc{PathObjects}.

\paragraph{Entry Point Suggestion} The detection of suitable entry points is still largely a matter of trial and error.
With the current implementation, developers can browse the source code, and test cases that cover a method are displayed alongside.
But the sole indication whether a unit test actually includes the scenario the developer wishes to analyze is its name.
In other words, the selection of suitable test cases either requires expertise or is a game of chance.
Consequently, assisting developers in finding proper entry points could be a beneficial task.
For instance, developers could be allowed to search for test cases with simple criteria, like definitions of objects or message sequences that have to occur in a unit test.
On a more abstract level, one could provide tools that allow to formulate requirements, such as "list test cases where an instance of the analog clock class acts as observer of a clock timer".
Accordingly, further research could tackle the questions are if entry point selection actually is a significant issue in practice, how such abstract semantic definitions can be matched to static source code, and to which extent this could help developers.